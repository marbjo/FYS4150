\documentclass{emulateapj}
%\documentclass[12pt,preprint]{aastex}

\usepackage{graphicx}
\usepackage{float}
\usepackage{amsmath}
\usepackage{epsfig,floatflt}
\usepackage{hyperref}
\usepackage[toc, page]{appendix}
\usepackage{verbatim, amsmath, amsfonts, amssymb, amsthm}
\usepackage[utf8]{inputenc}
\usepackage{textcomp}
\usepackage{float}
\usepackage{xcolor}
\usepackage{color}
\usepackage{listings}
\usepackage{fancyhdr}
\usepackage[T1]{fontenc}
\usepackage{url}
\usepackage[export]{adjustbox}
\usepackage{calc}

\usepackage{accents}
\newcommand{\dbtilde}[1]{\accentset{\approx}{#1}}
\newcommand{\vardbtilde}[1]{\tilde{\raisebox{0pt}[0.85\height]{$\tilde{#1}$}}}

\usepackage{lipsum}
\usepackage[para]{footmisc}


\definecolor{mygreen}{rgb}{0,0.6,0}
\definecolor{mygray}{rgb}{0.5,0.5,0.5}
\definecolor{mymauve}{rgb}{0.58,0,0.82}

\lstset{ %
	backgroundcolor=\color{white}\ttfamily\tiny,   % choose the background color; you must add \usepackage{color} or \usepackage{xcolor}; should come as last argument
	basicstyle=\tiny,        % the size of the fonts that are used for the code \footnotesize,
	breakatwhitespace=false,         % sets if automatic breaks should only happen at whitespace
	columns=fullflexible,    %no spaces between columns
	keepspaces=true,
	breaklines=true,                 % sets automatic line breaking
	breakatwhitespace=true,
	captionpos=b,                    % sets the caption-position to bottom
	commentstyle=\color{mygreen},    % comment style
	deletekeywords={...},            % if you want to delete keywords from the given language
	escapeinside={\%*}{*)},          % if you want to add LaTeX within your code
	extendedchars=true,              % lets you use non-ASCII characters; for 8-bits encodings only, does not work with UTF-8
	frame=single,	                   % adds a frame around the code
	keepspaces=true,                 % keeps spaces in text, useful for keeping indentation of code (possibly needs columns=flexible)
	keywordstyle=\color{blue},       % keyword style
	language=Python,                 % the language of the code
	morekeywords={*,...},           % if you want to add more keywords to the set
	%numbers=left,                    % where to put the line-numbers; possible values are (none, left, right)
	%numbersep=5pt,                   % how far the line-numbers are from the code
	%numberstyle=\tiny\color{mygray}, % the style that is used for the line-numbers
	rulecolor=\color{black},         % if not set, the frame-color may be changed on line-breaks within not-black text (e.g. comments (green here))
	showspaces=false,                % show spaces everywhere adding particular underscores; it overrides 'showstringspaces'
	showstringspaces=false,          % underline spaces within strings only
	showtabs=false,                  % show tabs within strings adding particular underscores
	stepnumber=1,                    % the step between two line-numbers. If it's 1, each line will be numbered
	stringstyle=\color{mymauve},     % string literal style
	tabsize=1,	                   % sets default tabsize to 2 spaces
	%title=\lstname                   % show the filename of files included with \lstinputlisting; also try caption instead of title
}

\begin{document}

\title{Numerical integration - An empirical study of Gaussian quadrature and Monte Carlo}

\author{Bruce Chappell and Markus Bjørklund}

\email{markus.bjorklund@astro.uio.no}

\altaffiltext{1}{Institute of Theoretical Astrophysics, University of
  Oslo, P.O.\ Box 1029 Blindern, N-0315 Oslo, Norway}

\begin{abstract}

Briefly what we are doing with what methods, and summary of the main results.

\end{abstract}
\keywords{Numerical integration --- Guassian quadrature --- Monte Carlo methods}

\section{Introduction}
\label{sec:introduction}
Brief overview of motivation, what problems we are actually solving, and overview of what is in the following sections.


\section{Theory}
\label{sec:method}


\section{Methods}
\label{sec:methods}
This section will describe the methods we utilized to acquire our results.

\subsection{Code}


\subsection{Algorithm}

\subsection{Parameter Selection}

\subsection{Unit Tests}


\section{Results}
\label{sec:results}


\section{Discussion}
\label{sec:discussion}


\section{Conclusions}
\label{sec:conclusions}



\subsection{Further research}


\newpage
\begin{thebibliography}{}

\bibitem{Lecture}[(Hjorth-Jensen, 2017)]{MHJ} Hjorth-Jensen, Morten \, Aug 23 2017, "Computational Physics Lectures:Numerical integration, from Newton-Cotes quadrature to Gaussian quadrature" , \url{http://compphysics.github.io/ComputationalPhysics/doc/pub/integrate/pdf/integrate-print.pdf}

\bibitem{Lecture}[(Hjorth-Jensen, 2019)]{MHJ} Hjorth-Jensen, Morten \, Oct 4 2019, "Computational Physics Lectures: Introduction to Monte Carlo methods" , \url{http://compphysics.github.io/ComputationalPhysics/doc/pub/mcint/pdf/mcint-print.pdf}

\bibitem{Project}[(Hjorth-Jensen, 2019)]{MHJ} Hjorth-Jensen, Morten \, Oct 2019, "Project 3"
\url{http://compphysics.github.io/ComputationalPhysics/doc/Projects/2019/Project3/pdf/Project3.pdf}

\end{thebibliography}

\section{Appendix}
All source code, data and figures can be found at the github repository: \url{https://github.com/marbjo/FYS4150/tree/master/Project3}

\end{document}